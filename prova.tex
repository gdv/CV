% Copyright 2013 Christophe-Marie Duquesne <chmd@chmd.fr>
% Copyright 2014 Mark Szepieniec <http://github.com/mszep>
% 
% ConText style for making a resume with pandoc. Inspired by moderncv.
% 
% This CSS document is delivered to you under the CC BY-SA 3.0 License.
% https://creativecommons.org/licenses/by-sa/3.0/deed.en_US

\startmode[*mkii]
  \enableregime[utf-8]  
  \setupcolors[state=start]
\stopmode

\setupcolor[hex]
\definecolor[titlegrey][h=757575]
\definecolor[sectioncolor][h=397249]
\definecolor[rulecolor][h=9cb770]

% Enable hyperlinks
\setupinteraction[state=start, color=sectioncolor]

\setuppapersize [A4][A4]
\setuplayout    [width=middle, height=middle,
                 backspace=20mm, cutspace=0mm,
                 topspace=10mm, bottomspace=20mm,
                 header=0mm, footer=0mm]

%\setuppagenumbering[location={footer,center}]

\setupbodyfont[11pt, helvetica]

\setupwhitespace[medium]

\setupblackrules[width=31mm, color=rulecolor]

\setuphead[chapter]      [style=\tfd]
\setuphead[section]      [style=\tfd\bf, color=titlegrey, align=middle]
\setuphead[subsection]   [style=\tfb\bf, color=sectioncolor, align=right,
                          before={\leavevmode\blackrule\hspace}]
\setuphead[subsubsection][style=\bf]

\setuphead[chapter, section, subsection, subsubsection][number=no]

%\setupdescriptions[width=10mm]

\definedescription
  [description]
  [headstyle=bold, style=normal,
   location=hanging, width=18mm, distance=14mm, margin=0cm]

\setupitemize[autointro, packed]    % prevent orphan list intro
\setupitemize[indentnext=no]

\setupfloat[figure][default={here,nonumber}]
\setupfloat[table][default={here,nonumber}]

\setuptables[textwidth=max, HL=none]

\setupthinrules[width=15em] % width of horizontal rules

\setupdelimitedtext
  [blockquote]
  [before={\setupalign[middle]},
   indentnext=no,
  ]


\starttext

\section[title={DELLA VEDOVA GIANLUCA --
CV},reference={della-vedova-gianluca-cv}]

\thinrule

\startblockquote
Professore Associato-Confermato (SSD INF/01 - Informatica)

Dipartimento di Informatica, Sistemistica e Comunicazione

Università degli Studi di Milano-Bicocca
\stopblockquote

\thinrule

\subsection[title={Ruoli},reference={ruoli}]

\startdescription{10/2012-oggi}
  Professore Associato, Dip. Informatica, Sistemistica e Comunicazioni,
  Università degli Studi di Milano--Bicocca.
\stopdescription

\startdescription{10/2005-09/2012}
  Professore Associato, Facoltà di Scienze Statistiche, Università degli
  Studi di Milano--Bicocca.
\stopdescription

\startdescription{05/2001-09/2005}
  Ricercatore, Facoltà di Scienze Statistiche, Università degli Studi di
  Milano--Bicocca.
\stopdescription

\subsection[title={Abilitazione},reference={abilitazione}]

\startdescription{08/2018-oggi}
  Abilitazione a Professore di I fascia, settore concorsuale 01/B1.
\stopdescription

\subsection[title={Titoli di studio},reference={titoli-di-studio}]

\startdescription{2001}
  {\bf Dottorato di Ricerca in Informatica}; Università degli Studi di
  Milano
\stopdescription

Tesi: \quotation{Multiple Sequence Alignment and Phylogenetic
Reconstruction: Theory and Methods in Biological Data Analysis.}
Advisors: Prof.~Tao Jiang (UC Riverside), Prof.~Giancarlo Mauri (Univ.
Milano-Bicocca), Prof.~Paola Bonizzoni (Univ. Milano-Bicocca).

\startdescription{1995}
  {\bf Laurea in Scienze dell'Informazione}; Università degli Studi di
  Milano.
\stopdescription

Tesi: Algoritmi sequenziali e paralleli per la decomposizione di grafi.

\section[title={Progetti di ricerca},reference={progetti-di-ricerca}]

\subsection[title={Responsabile},reference={responsabile}]

\startdescription{2020-2023}
  H2020-MSCA-RISE-2019 (importo Univ. Milano-Bicocca 197 800€)
  nell'ambito dell'azione Research Innovation Staff Exchange. Pan-genome
  Graph Algorithms and Data Integration (PANGAIA). In questo progetto,
  con coordinatore interno al mio laboratorio di ricerca, sono stato fra
  i principali estensori della proposta e sono il {\bf responsabile del
  WP5} Communication and Dissemination. Il progetto prevede la
  collaborazione fra 7 istituzioni beneficiarie (4 Università, 1 ente di
  ricerca e 2 aziende) di 6 nazioni europee e 4 partner in USA, Canada e
  Giappone.
\stopdescription

\startdescription{2021-2024}
  H2020-MSCA-ITN (importo Univ. Milano-Bicocca 261499,68€) nell'ambito
  dell'azione Innovative Training Network. ALgorithms for PAngenome
  Computational Analysis (ALPACA). In questo progetto ho la
  responsabilità di {\bf co-supervisionare un dottorando} nel periodo
  2021-2024. Il progetto vede 13 beneficiari (7 università, 5 enti di
  ricerca, 1 azienda) e 10 partner europei.
\stopdescription

\startdescription{2013-2016}
  Fondazione Cariplo 2013 Modulation of anti-cancer immune response by
  regulatory non-coding RNAs. Responsabile del WP bioinformatico. Il
  progetto mi ha portato ad essere {\bf responsabile Scientifico di due
  assegni di ricerca} annuali e ho co-supervisionato le attività di un
  terzo assegno di ricerca annuale.
\stopdescription

\startdescription{2011-2014}
  MIUR/Regione Lombardia 2011 (importo Univ. Milano-Bicocca 199 991€).
  Piattaforma di Analisi TRaslazionale Integrata. In questo progetto
  sono stato il {\bf responsabile di tutti gli aspetti bioinformatici}.
  Il progetto ha portato all'attivazione di un assegno di ricerca
  annuale (costo 23.000€) e 3 contratti di collaborazioni (ognuno di
  importo lordo al collaboratore fra 10.000€ e 12.000€). Ho
  co-supervisionato le attività relative all'assegno di ricerca, e sono
  stato unico responsabile scientifico delle attività relative ai
  contratti di collaborazione.
\stopdescription

\startdescription{2016}
  Fondo di Ateneo 2016 (12 490€). Modelli computazionali e algoritmi:
  aspetti teorici e sperimentali, con applicazioni alla Bioinformatica.
  {\bf Responsabile} del progetto.
\stopdescription

\startdescription{2015}
  Fondo di Ateneo 2015 (10 980€). Algoritmi combinatori e modelli di
  calcolo: aspetti teorici e applicazioni in Bioinformatica.
  {\bf Responsabile} del progetto.
\stopdescription

\startdescription{2014}
  Fondo di Ateneo 2014 (12 186€). Algoritmi e modelli computazionali:
  aspetti teorici e applicazioni nelle scienze della vita.
  {\bf Responsabile} del progetto.
\stopdescription

\startdescription{2013}
  Fondo di Ateneo 2013 (9 337€). Metodi algoritmici e modelli: aspetti
  teorici e applicazioni in bioinformatica. {\bf Responsabile} del
  progetto.
\stopdescription

\startdescription{2011}
  Fondo di Ateneo 2011 (4 055€). Tecniche algoritmiche avanzate in
  Biologia Computazionale. {\bf Responsabile} del progetto.
\stopdescription

\startdescription{2006}
  Grandi Attrezzature 2006 (40000€). Laboratorio Virtuale
  Statistico-Territoriale. Questo progetto, condiviso fra il
  Dipartimento di Statistica e il Dipartimento di Sociologia, ha portato
  all'acquisto di 2 server per la fornitura di servizi di ricerca e
  didattica in ambiti di Sociologia del territorio e Statistica
  computazionale. {\bf Responsabile} del progetto.
\stopdescription

\startdescription{2005-2008}
  MIUR/PRIN 2005, Potenzialità e ottimizzazione delle banche dati
  automatizzate in epidemiologia. In questo progetto sono stato il
  {\bf responsabile di tutti gli aspetti bioinformatici}.
\stopdescription

\subsection[title={Partecipante},reference={partecipante}]

\startdescription{2013-2016}
  Regione Lombardia. SPAC3 - Servizi smart della nuova Pubblica
  amministrazione per la Citizen-Centricity in cloud.
\stopdescription

\startdescription{2011-2014}
  MIUR/PRIN 2011

  Automi e Linguaggi Formali: Aspetti Matematici e Applicativi
\stopdescription

\startdescription{2003}
  MIUR/FIRB 2003. Bioinformatica per la Genomica e la Proteomica.
\stopdescription

\startdescription{2000-2001}
  NSF CCR-9988353, ITR-0085910.
\stopdescription

\startdescription{1999-2001}
  MURST COFIN 98 \quotation{Bioinformatica e ricerca genomica.}
\stopdescription

\startdescription{1994-1995}
  MURST 40\letterpercent{} \quotation{Algoritmi e strutture di calcolo.}
\stopdescription

\startdescription{1994-1995}
  ESPRIT-BRA ASMICS 2 n.~6317.
\stopdescription

\section[title={Ricerca},reference={ricerca}]

Ho pubblicato oltre 30 articoli su rivista scientifica con più di 1500
citazioni e h-index 17 (secondo Google Scholar). In particolare, 7
lavori hanno superato le 100 citazioni (fonte Google Scholar). Inoltre
ho lavorato con 166 coautori (fonte Scopus).

Dal 2016 al 2019 sono stato responsabile del Laboratorio di Ricerca
\quotation{AlgoLab - Experimental Algorithmics Lab,} DIpartimento di
Informatica, Sistemistica e Comunicazione, Università di Milano-Bicocca.
Il laboratorio di ricerca si è caratterizzata per numerose
collaborazioni internazionali. Il laboratorio si occupa del disegno di
algoritmi, della loro implementazione e dell'analisi sperimentale su
dataset di grandi dimensioni, pertanto l'ambito di ricerca riguarda sia
aspetti metodologici che sperimentali: ciò ha portato alla realizzazione
di vari strumenti software per risolvere vari problemi in
bioinformatica.

La mia ricerca si è sempre focalizzata sullo sviluppo di algoritmi
combinatori in Bioinformatica, con una forte componente fondazionale
basata sullo studio delle proprietà formali dei problemi computazionali
e passando per l'implementazione degli approcci proposti ed una
validazione degli stessi sia da un punto di vista teorico che
sperimentale.

Le principali tematiche investigate sono state:

{\bf Confronto di sequenze}

Il confronto di sequenze è uno dei problemi fondamentali in
Bioinformatica, in quanto sequenze simili corrispondono a parti di
genoma con funzionalità simili. In questo campo ha ottenuto importanti
risultati sulla complessità computazionale di alcune formulazioni del
problema, disegnando algoritmi efficienti per il calcolo di soluzioni
approssimate.

Le attività in questo ambito sono state svolte anche nell'ambito delle
collaborazioni scientifiche con il Winfried Just (Dept. Mathematics,
Ohio Univ.), Stéphane Vialette (LRI, Univ. Paris-Sud), Guillame Fertin
(LINA, Univ. Nantes).

{\bf Ricostruzione storie evolutive}

Il problema di ricostruire la storia evolutiva a partire da dati
genomici di specie o cellule esistenti è stato uno dei principali ambiti
della mia attività di ricerca. Inizialmente mi sono focalizzato su
problematiche classiche di confronto di alberi evolutivi, ottenendo
risultati sulla complessità di approssimazione, per poi descrivere
algoritmi per la riconciliazione di alberi di geni e di specie (per
individuare una storia evolutiva comune che sia una interpretazione
ammissibile di alberi apparentemente incompatibili). Più recente ho
contribuito a disegnare approcci efficienti per il problema della
ricostruzione di storie evolutive tumorali, anche supervisionando le
attività del gruppo di ricerca, oltre all'implementazione e l'analisi
sperimentali di tali approcci.

Questa tematica di ricerca è stata svolta anche nell'ambito delle
collaborazioni internazionali con Harold Todd Wareham (Dept. Computer
Science, Memorial Univ. Newfoundlands), Tao Jiang (Dept. Computer
Science, Univ. California at Riverside), Gabriella Trucco (Univ.
Milano), Jesper Jansson (The Hong Kong Polytechnic University), Iman
Hajirasouliha (Weill Cornell Medical College, New York). Responsabile
della collaborazione scientifica con il prof. Vladimir Filipovic, Univ.
Belgrado, che è in sabbatico presso l'Univ. Milano-Bicocca dal 20/1/18.
La collaborazione riguarda lo sviluppo di metaeuristiche in
Bioinformatica, ed in particolare per la ricostruzione di evoluzioni
tumorali.

{\bf Ricostruzioni di aplotipi}

Diverse specie, incluso l'uomo, sono diploidi: ogni cromosoma è presente
in due copie distinte detti aplotipi. Le tecnologie attuali riescono
solo con difficoltà a determinare l'aplotipo di provenienza. In questo
ambito, approcci informatici sono essenziali per riuscere ad ottenere i
singoli aplotipi ad un costo accettabile. Dopo avere ottenuti alcuni
risultati sulla complessità computazionale del problema, mi sono
dedicato al disegno di approcci euristici efficienti, seguendo anche gli
aspetti di implementazione e di analisi sperimentale. Il codice prodotto
è stato incorporato in uno dei principali programmi utilizzati dalla
comunità scientifica
(https://whatshap.readthedocs.io/en/latest/howtocite.html).

Le attività in questo ambito sono state svolte anche nell'ambito delle
collaborazioni scientifiche con Tao Jiang (Univ. California at
Riverside), Alessandra Stella (CNR), Tobias Marschall (Heinrich Heine
Univ., Düsseldorf), Romeo Rizzi (Univ. Verona).

{\bf Splicing alternativo}

Lo splicing alternativo è il fenomeno biologico che permette ad un
singolo gene di esprime più proteine. Dal punto di vista computazionale,
il problema principale è che non abbiamo in input l'intera sequenza che
viene tradotta in proteine (detta trascritto), ma solo sue porzioni.
Pertanto diventa necessario esaminare queste porzioni per costruire
tutti i trascritti, tenendo conto che presentano ripetizioni ed errori.
In questo ambito ho contributito a disegnare approcci efficienti,
seguendo anche gli aspetti di implementazione e di analisi sperimentale.

Le attività in questo ambito sono state svolte anche nell'ambito delle
collaborazioni scientifiche con Graziano Pesole (CNR e Head of Node di
Elixir Italy IIB), Ernesto Picardi (Univ. Bari).

{\bf Algoritmi su Grafi}

Gli aspetti combinatoriali di teoria dei grafi sono un ambito dove ho
contribuito allo sviluppo di algoritmi efficienti, principalmente per il
problema della decomposizione modulare, che è una tecnica che permette
il disegno di algoritmi efficienti per diversi problemi. Più
recentemente ho approfondito aspetti di teoria dei grafi, riuscendo a
sfruttarli in diversi ambiti della Bioinformatica, sia per ottenere
modelli computazionali più aderenti alla realtà biologica, sia per
ottenere approcci efficienti. Ciò è stato fondamentale nell'attività
inerente al settore innovativo della Pangenomica Computazionali, dove si
intendono studiare insiemi di genomi tramite strutture a grafo. Tale
ambito è l'oggetto dei progetti di ricerca europei PANGAIA (MSCA RISE
2019) e ALPACA (MSCA ITN 2020) indicati in precedenza.

Le attività in questo ambito sono state svolte anche nell'ambito delle
collaborazioni scientifiche con Alexander Schönhuth (Univ. Bielefeld).

{\bf Clustering}

Il clustering è uno dei problemi più studiati in Informatica, anche a
causa delle sue molteplici varianti ed applicazioni pratiche. Oltre ad
avere contribuito all'avanzamento delle conoscenze relative alla
complessità computazionale del problema del confronto di
cluterizzazioni, ho disegnato algoritmi efficienti per la
clusterizzazione di fingerprint, problema che trova la sua motivazione
nello studio di comunità microbiche. Inoltre ho contribuito a risultati
sulla complessità computazionale sul problema di anonimizzare tabelle
tramite l'omissione di dati, un problema che trova la sua motivazione
nell'ambito della data privacy.

\section[title={Didattica},reference={didattica}]

\subsection[title={Attività didattiche},reference={attività-didattiche}]

La mia attività didattica è iniziata come Ricercatore Universitario
presso la Facoltà di Scienze Statistiche nel 2001. Poichè in Facoltà non
erano presenti altri docenti dell'area Informatica, una parte
fondamentale della mia attività è stata la costruzione di nuovi
insegnamenti pensati per coorti di studenti con buone competenze
matematiche, ma non all'interno di Corsi di Laurea della classe
Informatica. Questa attività si è ripetuta più volte, sia in seguito
all'evoluzione dei corsi di studio di area statistica, che
successivamente all'interno della laurea triennale in Informatica e
della laurea magistrale in Data Science.

La mia modalità di insegnamento è centrata sulla metodologia di active
learning, che richiede una forte e continua interazione fra docente e
studente, oltre che fra studenti. Ciò implica normalmente la costruzione
di problemi che gli studenti devono affrontare, da soli o in gruppo.
Anche in questo caso l'attività progettuale del corso è innovativa, in
quanto le pratiche preesistenti e il materiale didattico erano invece
pensate per modalità più tradizionali, dove gli studenti avevano
principalmente un ruolo di uditori e la componente di problem solving
era confinata ad un ruolo secondario.

\subsection[title={Corsi di dottorato di
ricerca},reference={corsi-di-dottorato-di-ricerca}]

\startdescription{2016, 2018, 2020}
  \quotation{Advanced Algoriths,} Dottorato di Ricerca in Informatica,
  Univ. Milano-Bicocca (20 ore). In questo caso ho dovuto progettare
  completamente l'insegnamento che non era mai stato erogato in
  precedenza.
\stopdescription

\subsection[title={Titolarità di Insegnamenti in Corsi di
Studio},reference={titolarità-di-insegnamenti-in-corsi-di-studio}]

\startdescription{2017-oggi}
  \quotation{Foundations of Computer Science,} Laurea Magistrale in Data
  Science, Univ. Milano-Bicocca. (6 CFU). In questo caso ho dovuto
  progettare completamente l'insegnamento, in quanto il corso di studio
  era di nuova attivazione.
\stopdescription

\startdescription{2014-oggi}
  \quotation{Elementi di Bioinformatica,} Laurea Triennale in
  Informatica, Univ. Milano-Bicocca (8 CFU). In questo caso ho dovuto
  progettare completamente l'insegnamento che non era mai stato erogato
  in precedenza.
\stopdescription

\startdescription{2009-2020}
  \quotation{Basi di Dati,} Laurea Triennale in Statistica e Gestione
  delle Informazioni, Laurea Triennale in Scienze Statistiche ed
  Economiche, Univ. Milano-Bicocca (6 CFU).
\stopdescription

\startdescription{2007-oggi}
  \quotation{Bioinformatica,} Laurea Magistrale in Biostatistica, Univ.
  Milano-Bicocca (6 CFU). In questo caso ho dovuto progettare
  completamente l'insegnamento che non era mai stato erogato in
  precedenza.
\stopdescription

\startdescription{2010-2013}
  \quotation{Algoritmi su stringhe,} Laurea Triennale in Informatica,
  Univ. Milano-Bicocca. In questo caso ho dovuto progettare
  completamente l'insegnamento che non era mai stato erogato in
  precedenza.
\stopdescription

\startdescription{2008}
  \quotation{Strumenti informatici per la statistica M,} Laurea
  Magistrale in Biostatistica, blended e-learning, Univ. Milano-Bicocca.
  In questo caso ho dovuto progettare completamente l'insegnamento che
  non era mai stato erogato in precedenza.
\stopdescription

\startdescription{2007-2009}
  \quotation{Informatica Applicata S,} Laurea Specialistica in
  Biostatistica (2 CFU), Univ. Milano-Bicocca. In questo caso ho dovuto
  progettare completamente l'insegnamento che non era mai stato erogato
  in precedenza.
\stopdescription

\startdescription{2001-2008}
  \quotation{Laboratorio Statistico-Informatico,} tutte le Lauree
  triennali della Facoltà di Scienze Statistiche (6 CFU), Univ.
  Milano-Bicocca. In questo caso ho dovuto riprogettare completamente
  l'insegnamento che era stato erogato in precedenza una sola volta, da
  un docente esterno non afferente ad alcuna Università.
\stopdescription

\startdescription{2001-2008}
  \quotation{Programmazione e Basi Dati,} tutte le Lauree triennali
  della Facoltà di Scienze Statistiche (6 CFU), Univ. Milano-Bicocca. In
  questo caso ho dovuto riprogettare completamente l'insegnamento che
  era mai stato erogato in precedenza. I contenuti sono stati
  nell'ambito di Basi di Dati (il corso ha poi cambiato nome in Basi di
  Dati) e in Italia è stato il primo insegnamento di Basi di Dati
  dedicato a studenti nell'area Statistica.
\stopdescription

\startdescription{2008}
  \quotation{Fondamenti di Informatica,} Laurea Magistrale in
  Biostatistica, Univ. Milano-Bicocca. In questo caso ho dovuto
  progettare completamente l'insegnamento che non era mai stato erogato
  in precedenza.
\stopdescription

\startdescription{2006}
  \quotation{Laboratorio di Informatica,} tutte le Lauree triennali
  della Facoltà di Scienze Statistiche (6 CFU), Univ. Milano-Bicocca.
\stopdescription

\subsection[title={Esercitatore di
insegnamenti},reference={esercitatore-di-insegnamenti}]

\startdescription{2004-2005}
  \quotation{Bioinformatica: tecniche di base (laboratorio),} Laurea
  Magistrale in Bioinformatica (2 CFU), Univ. Milano-Bicocca.
\stopdescription

Insegnamenti di Master Universitari

\startdescription{2003}
  \quotation{Fondamenti di Informatica e Elementi di Programmazione}
  (Fundamentals of Computer Science and Elements of Programming), Master
  di primo livello in Bioinformatica, Univ. Milano-Bicocca.
\stopdescription

\startdescription{2001-2002}
  \quotation{Sistemi Informatici e Elementi di Programmazione,} Master
  di primo livello in Bioinformatica, Univ. Milano-Bicocca.
\stopdescription

\startdescription{2012, 2014}
  “Data Base e Sistemi Informativi, Master di primo livello in
  Amministratore di Sistema per la Diagnostica per Immagini, Univ.
  Milano-Bicocca.
\stopdescription

\startdescription{2007, 2010, 2011}
  \quotation{Fondamenti di Informatica,} Master di primo livello in
  Amministratore di Sistema per la Diagnostica per Immagini, Univ.
  Milano-Bicocca.
\stopdescription

\section[title={Supervisione},reference={supervisione}]

\subsection[title={Assegnisti di
ricerca},reference={assegnisti-di-ricerca}]

Sono stato {\bf responsabile scientifico} dei seguenti assegni di
ricerca:

\startdescription{2018-2019}
  Murray Patterson. Assegno di ricerca biennale con argomento
  \quotation{Haplotype assembly from sequencing reads.} Attualmente
  Assistant Professor (Tenure-track) presso Georgia State University.
\stopdescription

\startdescription{2015}
  Hassan Mahmoud Mohamed Ramadan Mohamed. Assegno di ricerca annuale con
  argomento \quotation{Methodology for treatment and data analysis of
  NGS data for the detection of alternate splicing events}
\stopdescription

Inoltre ho collaborato alla formazione dei seguenti assegnisti di
ricerca:

\startitemize[n,packed][stopper=.]
\item
  Marco Previtali
\item
  Stefano Beretta
\stopitemize

\subsection[title={Dottorandi},reference={dottorandi}]

Sono stato {\bf supervisor} delle seguenti tesi di dottorato in
Informatica

\startitemize[n][stopper=.]
\item
  Riccardo Dondi, \quotation{Computational Problems in the Study of
  Genomic Variations,} 2004 (attualmente Prof.~Associato presso
  l'Università di Bergamo)
\item
  Stefano Beretta, \quotation{Algorithms for Next Generation Sequencing
  Data Analysis,} 2012 (attualmente Bionformatico presso San Raffaele
  Telethon Institute for Gene Therapy)
\item
  Marco Previtali, \quotation{Self-indexing for de novo assembly,} 2017
\item
  Simone Ciccolella, \quotation{Algorithms for cancer phylogeny
  inference,} termine previsto 2021
\stopitemize

Inoltre ho collaborato attivamente alla formazione dei seguenti
dottorandi:

\startitemize[n,packed][stopper=.]
\item
  Anna Paola Carrieri, attualmente Research Staff Member presso IBM
  Research Lab, UK
\item
  Simone Zaccaria, attualmente Group Leader presso Department of
  Oncology, Univ. College London
\item
  Giulia Bernardini, attualmente postdoc press CWI, Amsterdam.
\stopitemize

\subsection[title={Studenti di laurea
magistrale},reference={studenti-di-laurea-magistrale}]

Sono stato relatore o correlatore di oltre 10 studenti di laurea
magistrale in Informatica, in Scienze Statistiche ed Economiche, in
Biostatistica, in Data Science.

In particolare ho seguito le attività nell'ambito del programma
{\bf Exchange mobility EXTRA UE} dei seguenti studenti:

\startitemize[n,packed][stopper=.]
\item
  Ramesh Rajaby che ha trascorso 6 mesi presso il gruppo di ricerca del
  prof. Jesper Jansson (Kyoto University, Giappone). Ramesh Rajaby è
  attualmente postdoc alla National University of Singapore.
\item
  Simone Ciccolella che ha trascorso un periodo di 3 mesi presso il
  gruppo di ricerca del prof. Iman Hajirasouliha (Weill Cornell Medical
  College, New York). Simone Ciccolella è attualmente dottorando sotto
  la mia supervisione.
\stopitemize

\subsection[title={Studenti di laurea
triennale},reference={studenti-di-laurea-triennale}]

Ho supervisionato le attività di stage e di prova finale di oltre 40
studenti di Laurea Triennale in Statistica e Gestione delle
Informazioni, in Informatica, in Scienze Statistiche ed Economiche, in
Statistica.

\section[title={Attività di servizio},reference={attività-di-servizio}]

\subsection[title={Servizi per
l'Ateneo},reference={servizi-per-lateneo}]

\startdescription{2020-oggi}
  {\bf Rappresentante dell'Università degli Studi di Milano-Bicocca}
  nell'Assemblea Generale della Joint Research Unit ELIXIR IIB, nodo
  italiano di Elixir Europe, l'organizzazione intergovernativa per la
  Bioinformatica.
\stopdescription

\startdescription{2019-oggi}
  {\bf Vice coordinatore del Dottorato} di Ricerca in Informatica, Univ.
  Milano - Bicocca
\stopdescription

\startdescription{2018-oggi}
  {\bf Assicuratore di Qualità} del Corso di Laurea Magistrale in Data
  Science.
\stopdescription

\startdescription{2019-oggi}
  {\bf Presidente della Commissione Didattica} del Corso di Laurea
  Magistrale in Data Science. (check data)
\stopdescription

\startdescription{2018-oggi}
  {\bf Responsabile} per l'Università degli Studi di Milano-Bicocca
  della Convenzione quadro con l'Istituto Nazionale di Genetica
  Molecolare.
\stopdescription

\startdescription{2020-oggi}
  {\bf Membro del Comitato Scientifico} del Master di secondo livello
  \quotation{qOmics: quantitative methods for Omics Data,} Univ. di
  Milano-Bicocca e Univ. di Pavia.
\stopdescription

\startdescription{2018-oggi}
  membro della {\bf Commissione Paritetica Docenti-Studenti} del
  Dipartimento di Informatica, Sistemistica e Comunicazione
\stopdescription

\startdescription{2013-oggi}
  Membro del collegio docenti del Dottorato di Ricerca in Informatica,
  Univ. Milano - Bicocca
\stopdescription

\startdescription{2015-2020}
  Coordinatore Tecnico Locale per l'Univ. Milano-Bicocca delle attività
  inerenti Elixir IIB
\stopdescription

\startdescription{2016-oggi}
  {\bf Responsabile delle attività di training} per giovani ricercatori
  nell'ambito delle iniziative di Elixir IIB. In particolare ho
  organizzato corsi su \quotation{Genome Assembly and Annotation,}
  \quotation{Data Carpentry Workshop,} \quotation{Software Carpentry
  Workshop,} \quotation{Docker Advanced,} \quotation{Exome analysis with
  Galaxy.}
\stopdescription

\startdescription{2016-2018}
  Membro della commissione orientamento del Dipartimento di Informatica,
  Sistemistica e Comunicazione
\stopdescription

\startdescription{2002-2012}
  {\bf Referente dell'area Informatica} all'interno della Facoltà di
  Scienze Statistiche. L'incarico è terminato con lo scioglimento della
  Facoltà.
\stopdescription

\startdescription{2004-2012}
  {\bf Rappresentante della Facoltà} di Scienze Statistiche all'interno
  del comitato di Ateneo per l'Informatica.
\stopdescription

\startdescription{2010-2012}
  {\bf Delegato del Preside} della Facoltà di Scienze Statistiche per
  l'e-learning.
\stopdescription

\startdescription{2007-2012}
  {\bf Responsabile} di tutti i laboratori informatici della Facoltà di
  Scienze Statistiche
\stopdescription

\startdescription{2010-2012}
  Membro della commissione per l'elearning della Facoltà di Scienze
  Statistiche.
\stopdescription

\startdescription{2002-2004}
  {\bf Referente dell'area Informatica} nella commissione della Facoltà
  di Scienze Statistiche che ha disegnato il corso di Laurea Triennale
  in Statistica e Gestione delle Informazioni. Il CdL è ancora attivo.
\stopdescription

\startdescription{2004}
  {\bf Responsabile} per la Facoltà di Scienze Statistiche del corso
  \quotation{Information Technology For Problem Solving (IT4PS),}
  organizzato dalla Fondazione CRUI.
\stopdescription

\startdescription{2003}
  {\bf Responsabile} del corso 167388 \quotation{Laboratorio
  Complementare di Informatica per Statistici,} all'interno del progetto
  FSE 156165 \quotation{Progetto Quadro Università degli Studi di
  Milano-Bicocca.} Il corso ha introdotto competenze informatiche di
  livello intermedio a studenti della Facoltà di Scienze Statistiche.
  L'iniziativa è stata interamente finanziata dall'Unione Europea.
\stopdescription

\subsection[title={Servizi per la comunità
scientifica},reference={servizi-per-la-comunità-scientifica}]

\startdescription{2013-oggi}
  Membro del {\bf Program Commitee} delle seguenti conferenze
  scientifiche: Computability in Europe (CiE2013, CiE2019, CiE 2020),
  Workshop on Algorithms in Bioinformatics (WABI 2020), ISCB European
  Conference on Computational Biology (ECCB 2019), Combinatorial Pattern
  Matching (CPM 2019), Symposium on String Processing and Information
  Retrieval (SPIRE 2017), Bioinformatics Open Source Conference (BOSC
  2016-2019).
\stopdescription

\startdescription{2018-oggi}
  Executive Officer dello {\bf Steering Committee} della conferenza
  Computability in Europe
\stopdescription

\startdescription{1997-oggi}
  Revisore di articoli per le seguenti riviste scientifiche: ACM/IEEE
  Transactions on Computational Biology and Bioinformatics,
  Algorithmica, Algorithms, Bioinformatics, Briefings in Bioinformatics,
  Graphs and Combinatorics, Information Processing Letters, INFORMS J.
  Computing, Journal of Computer Science and Technology, Theoretical
  Computer Science, Theory of Computing Systems.
\stopdescription

\startdescription{2020}
  {\bf Chair} del Workshop Data Structure in Bioinformatics (DSB 2020)
\stopdescription

\startdescription{2020}
  {\bf Editor} degli atti del convegno Computability in Europe 2020,
  LNCS 12098.
\stopdescription

\startdescription{2020}
  {\bf Organizzazione}, insieme con Iman Hajirasouliha (Weill-Cornell
  Medical College) della sessione speciale \quotation{Large Scale
  Bioinformatics and Computational Sciences} della conferenza
  Computability in Europe 2020
\stopdescription

\startdescription{2016-2019}
  Membro dell'{\bf Editorial Board} della rivista scientifica
  \quotation{Advances in Bioinformatics.}
\stopdescription

\startdescription{2018}
  Membro del comitato di valutazione della Tesi di Dottorato di Mattia
  Gastaldello, Univ. Roma la Sapienza.
\stopdescription

\startdescription{2018}
  Membro del comitato di valutazione della Tesi di Dottorato di Luca
  Ferrari, Univ. Milano.
\stopdescription

\startdescription{2017}
  Membro della {\bf commissione giudicatrice} in valutazione comparativa
  per una posizione di RTDb, Univ. Milano
\stopdescription

\startdescription{2004}
  {\bf editor di special issue} del Journal of Computer Science and
  Technology.
\stopdescription

\subsection[title={Relazioni a
convegni},reference={relazioni-a-convegni}]

Ho presentato i risultati della mia attività di ricerca a diverse
conferenze. In particolare:

\startitemize[packed]
\item
  Sono stato relatore alla conferenza internazionale Intelligent Systems
  for Molecular Biology (ISMB) 2001: si tratta della principale
  conferenza in Bioinformatica ({\bf CORE: A, Microsoft Academic: A+}).
  Ho presentato un lavoro in cui ero l'unico autore del mio Ateneo.
\item
  Sono stato {\bf invited speaker} della sessione speciale
  \quotation{Algorithmics for Biology} della conferenza Computability in
  Europe 2017.
\stopitemize

\section[title={Terza Missione},reference={terza-missione}]

\subsection[title={Incarichi da enti
esterni},reference={incarichi-da-enti-esterni}]

\startdescription{2003-2004}
  Incarico per studi relativi agli aspetti informatici relativi al
  progetto INTERREG IIIB (2000-2006) W.E.S.T. WOMEN EAST SMUGGLING
  TRAFFICKING (WP.2.2). {\bf Committente: Fondazione Ismu-Iniziative e
  Studi sulla Multietnicità}. Sono stato l'unico ricercatore informatico
  coinvolto nel progetto.
\stopdescription

\startdescription{2005}
  Incarico per studi relativi agli aspetti informatici relativi al
  progetto \quotation{Indagine Finalizzata all'Analisi degli Effetti
  Prodotti dai Processi di Regolarizzazione dei Lavoratori
  Extracomunitari, con Particolare Riferimento al Mercato del Lavoro e
  all'Integrazione Sociale nelle Regioni Ob. 1,} finanziato nell'ambito
  della Misura I.2 FESR \quotation{Adeguamento del Sistema di Controllo
  Tecnologico del Territorio} del PON Sicurezza 2000/2006.
  {\bf Committente: Fondazione Ismu-Iniziative e Studi sulla
  Multietnicità}. Sono stato l'unico ricercatore informatico coinvolto
  nel progetto.
\stopdescription

\section[title={Pubblicazioni},reference={pubblicazioni}]

\subsection[title={Articoli in riviste
internazionali},reference={articoli-in-riviste-internazionali}]

Bonizzoni et al. (2020)

\subsection[title={Articoli in atti di convegno
internazionali},reference={articoli-in-atti-di-convegno-internazionali}]

\thinrule

\startblockquote
\useURL[url1][mailto:gianluca.dellavedova@unimib.it][][gianluca.dellavedova@unimib.it]\from[url1]
• \useURL[url2][http://gianluca.dellavedova.org]\from[url2] +39 026448
7881 • viale Sarca 336, Milano(Italy)
\stopblockquote

\thinrule

Milano, 02/02/2021

F.to Gianluca Della Vedova

\startcslreferences

\reference[ref-Bonizzoni_2020]{}%
Bonizzoni, Paola, Gianluca Della Vedova, Yuri Pirola, Marco Previtali,
and Raffaella Rizzi. 2020. “Computing the Multi-String BWT and LCP Array
in External Memory.” {\em Theoretical Computer Science}, November.
\useURL[url3][https://doi.org/10.1016/j.tcs.2020.11.041]\from[url3].

\stopcslreferences

\stoptext
